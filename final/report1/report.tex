% Created 2017-03-01 Wed 20:38
\documentclass[10pt]{article}
\usepackage[utf8]{inputenc}
\usepackage[T1]{fontenc}
\usepackage{fixltx2e}
\usepackage{graphicx}
\usepackage{longtable}
\usepackage{float}
\usepackage{wrapfig}
\usepackage{rotating}
\usepackage[normalem]{ulem}
\usepackage{amsmath}
\usepackage{textcomp}
\usepackage{marvosym}
\usepackage{wasysym}
\usepackage{amssymb}
\tolerance=1000
\usepackage{natbib}
\usepackage[linktocpage,pdfstartview=FitH,colorlinks,
linkcolor=blue,anchorcolor=blue,
citecolor=blue,filecolor=blue,menucolor=blue,urlcolor=blue]{hyperref}
\usepackage[margin=2cm]{geometry}
\pagenumbering{gobble}
\usepackage{wrapfig}
\usepackage{multicol}
\setlength\columnsep{20pt}
\author{Alejandro Rodríguez Salamanca - r0650814 - Erasmus}
\date{}
\title{Nonlinear regression and classification with MLPs}
\begin{document}

\maketitle

\begin{multicols}{2}

  \section*{Regression}
  In this problem, the objective is to approximate a nonlinear function using
  a feedforward artificial neural network. The first step is to build the dataset.
  \subsection*{Define the dataset}
  First, 
  The dataset must be constructed using the digits of the student number. The
  largest five digits in descending order are used, this is, \textbf{86541}.
  
  

  \section*{Classification}
  Given my student number, r065081\textbf{4}, the dataset is constructed in the folliwing way:
  \begin{itemize}
    \item For the positive class, $C_{+} = 4$
    \item For the negative class, $C_{-} = 5+6$
  \end{itemize}
  

\end{multicols}

  \section*{Appendix}

\end{document}
